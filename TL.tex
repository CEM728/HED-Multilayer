\documentclass[11pt]{article}
% Horizontal Magnetic Dipole over a lossy half-space
\usepackage[utf8]{inputenc} % Use it to include other characters than ABC
\usepackage[cmex10]{amsmath}
\usepackage{mdwmath}
\usepackage{mdwtab}
\usepackage{hyperref}
\usepackage{physics} % For using the oridnary derivative nomenclature
\usepackage{datetime} % Insert date and time
\usepackage[letterpaper, margin=1in]{geometry}
\usepackage{graphicx}
\usepackage{pgfplots}
\usepackage{tikz}
\usepackage{standalone}
\usepackage[americanresistors,americaninductors]{circuitikz}
\usetikzlibrary{positioning}
\usetikzlibrary{arrows}
\usepackage{subfig}
% \usepackage{pdfsync} % use to back search tex lines from pdf
% pdflatex -synctex=-1
% \usepackage{mathptmx} % Times new Roman

% ------------------------------- Useful Tricks Learnt
% Use ={}& to align subequations to the left
% Use = for single equations
% Put comments % in between the lines in order to avoid forming a new paragraph.
% To enter special characters into Inkspace figures, use Ctrl+U and then enter       the unicode. e.g., for \times symbol, the unicode is U+0D7. So the key entry would be Ctrl+U U+0d7 and then press enter.
%
% ----------------- To compile with references use the following order in Shell"
% 1. pdflatex filename.tex
% 2. bibtex filename (no extension)
% 3. bibtex filename (no extension)
% 4. pdflatex filename.tex
% -----------------

% Personal definitions
% Operators
\renewcommand{\v}[1]{\mathbf{#1}} % vectors
\newcommand{\ti}[1]{\tilde{#1}} % spectral representation

% Symbols
\renewcommand{\O}{\omega}  % omega
\newcommand{\E}{\varepsilon}  % epsilon
\renewcommand{\u}{\mu}  % mu
\newcommand{\p}{\rho}  % rho
\newcommand{\x}{\times}  % times
\renewcommand{\inf}{\infty}  % infinity
\newcommand{\infint}{\int\limits_{-\inf}^\inf} % integral by R
\newcommand{\del}{\nabla}  % nabla operator
\renewcommand{\^}{\hat}  % unit vector


\begin{document}
  \title{\textsc{Horizontal Electric Dipole}\\}
  \date{\footnote{Last Modified: \currenttime, \today.}}
  \maketitle
  The transmission line Greens functions (TLGF) for planar multilayers of a High Electron Mobility Transistor (HEMT) are derived by modeling the radiating 2DEG as a embedded horizontal electric dipole illustrated in Fig. ().

  From Transmission Line (TL) theory, the voltage and current in a source-free section can be expressed through the telegrapher's equations:
  \begin{subequations}
    \begin{align}
      \dv{}{z}{V(z)} ={}& jk_z Z I(z),
      \label{eq:TL_V}\\
      \dv{}{z}{I(z)} ={}& jk_z Y V(z).
      \label{eq:TL_I}
    \end{align}
    \label{eq:TL}
  \end{subequations}

where $k_z$ is the propagation constant of the TL, $Z$ and $Y$ are respectively, the characteristic impedance and admittance of the transmission line. By combining (\ref{eq:TL_V}) and (\ref{eq:TL_I}), a second-order homogoneous differential equation in either $V$ or $I$ can be obtained:

\begin{subequations}
  \begin{align}
    \dv{}{z}{V(z)} ={}& jk_z Z I(z),
    \label{eq:TL_V}\\
    \dv{}{z}{I(z)} ={}& jk_z Y V(z).
    \label{eq:TL_I}
  \end{align}
  \label{eq:TL}
\end{subequations}





  \clearpage % Force Bibliography to the end of document
  \bibliography{mylib}
  \bibliographystyle{ieeetr}

\end{document}
