\documentclass[11pt]{article}
% Horizontal Magnetic Dipole over a lossy half-space
\usepackage[utf8]{inputenc} % Use it to include other characters than ABC
\usepackage[cmex10]{amsmath}
\usepackage{mdwmath}
\usepackage{mdwtab}
\usepackage{hyperref}
\usepackage{physics} % For using the oridnary derivative nomenclature
\usepackage{datetime} % Insert date and time
\usepackage[letterpaper, margin=1in]{geometry}
\usepackage{graphicx}
\usepackage{pgfplots}
\usepackage{tikz}
\usepackage{standalone}
\usepackage[americanresistors,americaninductors]{circuitikz} % For circuit elements
\usepackage{tikz-dimline} % For dimensional drawing
\usetikzlibrary{positioning}
\usetikzlibrary{arrows}
\usepackage{subfig}
% The following is done to hide ugly color boces around the links
\usepackage{xcolor}
\hypersetup{
    colorlinks,
    linkcolor={red!50!black},
    citecolor={blue!50!black},
    urlcolor={blue!80!black}
}
% \usepackage{pdfsync} % use to back search tex lines from pdf
% pdflatex -synctex=-1
% \usepackage{mathptmx} % Times new Roman

% ------------------------------- Useful Tricks Learnt
% Use ={}& to align subequations to the left
% Use = for single equations
% Put comments % in between the lines in order to avoid forming a new paragraph.
% To enter special characters into Inkspace figures, use Ctrl+U and then enter       the unicode. e.g., for \times symbol, the unicode is U+0D7. So the key entry would be Ctrl+U U+0d7 and then press enter.
%
% ----------------- To compile with references use the following order in Shell"
% 1. pdflatex filename.tex
% 2. bibtex filename (no extension)
% 3. bibtex filename (no extension)
% 4. pdflatex filename.tex
% -----------------

% Personal definitions
% Operators
\renewcommand{\v}[1]{\mathbf{#1}} % vectors
\newcommand{\ti}[1]{\tilde{#1}} % spectral representation

% Symbols
\renewcommand{\O}{\omega}  % omega
\newcommand{\E}{\varepsilon}  % epsilon
\renewcommand{\u}{\mu}  % mu
\newcommand{\p}{\rho}  % rho
\newcommand{\x}{\times}  % times
\renewcommand{\inf}{\infty}  % infinity
\newcommand{\infint}{\int\limits_{-\inf}^\inf} % integral by R
\newcommand{\del}{\nabla}  % nabla operator
\renewcommand{\^}{\hat}  % unit vector


\begin{document}
  \title{\textsc{Horizontal Electric Dipole}\\}
  \date{\footnote{Last Modified: \currenttime, \today.}}
  \maketitle
  The transmission line Greens functions (TLGF) for planar multilayers of a High Electron Mobility Transistor (HEMT) are derived by modeling the radiating 2DEG as a embedded horizontal electric dipole illustrated in Fig. ().

  From Transmission Line (TL) theory, the voltage and current in a source-free section can be expressed through the telegrapher's equations:
  \begin{subequations}
    \begin{align}
      \dv{}{z}{V(z)} ={}& -jk_z Z I(z),
      \label{eq:TL_V}\\
      %
      \dv{}{z}{I(z)} ={}& -jk_z Y V(z).
      \label{eq:TL_I}
    \end{align}
    \label{eq:TL}
  \end{subequations}

where $k_z$ is the propagation constant of the TL, $Z$ and $Y$ are respectively, the characteristic impedance and admittance of the transmission line. By combining (\ref{eq:TL_V}) and (\ref{eq:TL_I}), a second-order homogoneous differential equation in either $V$ or $I$ can be obtained:

\begin{align}
  \left( \dv[2]{}{z} + k_z^2 \right) \Bigg \{
  \begin{split}
    V(z)\\
    I(z)
  \end{split} = 0
    \label{eq:HomoVI}
\end{align}

A travelling wave solution of (\ref{eq:HomoVI}) can be written as:

\begin{align}
  \begin{split}
    V(z)\\
    Z I(z)
  \end{split} \Bigg \}= V^+ e^{-jk_z (z-z')} + V^- e^{jk_z (z-z')}
  \label{eq:homo_sol}
\end{align}

where the two terms represent forward and backward propagating waves with amplitudes $V^+$ and $V^-$ respectively. The exponentials can be phased referenced to any point; the current choice will become evident in the treatment of a multilayer case.

For a transmission line of infinite extent excited by a current source of unit amplitude located at $z = z'$, the voltage and current can be expressed as:

\begin{subequations}
  \begin{align}
    V(z) ={}& \frac{Z}{2} e^{-jk_z \left|z-z'\right| },
    \label{eq:TL_source_V}\\
    %
    I(z) ={}& \pm \frac{1}{2} e^{-jk_z \left|z-z'\right|}.
    \label{eq:TL_source_I}
  \end{align}
  \label{eq:TL_source}
\end{subequations}

The $+$ sign in (\ref{eq:TL_source_I}) is for $z > z'$ and vice versa. The discontinuity due to the presence of the source at $z = z'$ should also be observed due to change of signs at $ z = z'$.

\section{Multi-section Transmission Line}

We now consider a multi-section transmission line excited by a unit strength current source located in the $n^{th}$ section as illustrated in Fig. \ref{fig:TL}.

\begin{figure}[h!]
  \centering
  \includestandalone[width=1\textwidth]{figures/tl}
  \caption{Current Source in a multi-section transmission line}
  \label{fig:TL}
\end{figure}

The voltage and current expressions in any section now consist of a particular solution due to source and a homogeneous solution accounting for boundaries of the section. Due to a current source at $z = z'$, the solutions at any point $z$ in the $n^{th}$ section with propagation constant $k_{zn}$ become:

\begin{subequations}
  \begin{align}
    V_i(z, z') ={}& \frac{Z}{2} e^{-jk_z \left|z-z'\right| } + A_n e^{-j k_{zn} z } + B_n e^{j k_{zn} z},
    \label{eq:multi_V}\\
    %
    I_i(z, z') ={}& \pm \frac{1}{2} e^{-jk_z \left|z-z'\right|} + A_n e^{-j k_{zn} z } - B_n e^{j k_{zn} z}.
    \label{eq:multi_I}
  \end{align}
  \label{eq:multi}
\end{subequations}

In order to find the unknowns $A_n$ and $B_n$, we invoke the boundary conditions that ensure continuity of the characteristic impedance along the transmission line. At the left edge $z_n$ and the right edge $z_{n+1}$, we write:

\begin{subequations}
  \begin{align}
    \frac{V_i(z_n,z')}{I_i(z_n,z')} ={}& -\overleftarrow{Z_n} , \quad z = z_n
    \label{eq:BC_left}\\
    %
    \frac{V_i(z_{n+1},z')}{I_i(z_{n+1},z')} ={}& \overrightarrow{Z_n} ,\quad  z = z_{n+1}.
    \label{eq:BC_right}
  \end{align}
  \label{eq:BC}
\end{subequations}

From (\ref{eq:multi}) and (\ref{eq:BC}), we obtain:

\begin{subequations}
  \begin{align}
    \left( Z_n - \overleftarrow{Z_n} \right) e^{-j k_{zn} (z' - z_n) } +  A_n\left( Z_n + \overleftarrow{Z_n} \right) e^{-j k_{zn} z_n } +
    B_n\left( Z_n - \overleftarrow{Z_n} \right) e^{+j k_{zn} z_n } = 0
    \label{eq:BC_at_left}\\
    %
    \left( Z_n - \overrightarrow{Z_n} \right) e^{-j k_{zn} (z_{n+1} - z') } +  A_n\left( Z_n - \overrightarrow{Z_n} \right) e^{-j k_{zn} z_{n+1} } +
    B_n \left( Z_n - \overrightarrow{Z_n} \right) e^{+j k_{zn} z_{n+1} } = 0
    \label{eq:BC_at_right}
  \end{align}
  \label{eq:BC_val}
\end{subequations}

The solution of the system of two linear equations (\ref{eq:BC_val}) can be written in general form as:

\begin{align}
  A_n = \frac{ c e - b f }{bd - a e}\quad , \quad
  %
  B_n = \frac{ a f - c d }{bd - a e}
  \label{eq:gen_sol}
\end{align}

where,
\begin{subequations}
  \begin{align}
    a ={}& \left( Z_n + \overleftarrow{Z_n} \right) e^{-j k_{zn} z_n },\\
    b ={}& \left( Z_n - \overleftarrow{Z_n} \right) e^{+j k_{zn} z_n },\\
    c ={}& \left( Z_n - \overleftarrow{Z_n} \right) e^{-j k_{zn} (z' - z_n)},\\
    d ={}& \left( Z_n - \overrightarrow{Z_n} \right) e^{-j k_{zn} z_{n+1} },\\
    e ={}& \left( Z_n - \overrightarrow{Z_n} \right) e^{+j k_{zn} z_{n+1} },\\
    f ={}& \left( Z_n - \overrightarrow{Z_n} \right) e^{-j k_{zn} (z_{n+1} - z') }.
    \label{eq:terms}
  \end{align}
\end{subequations}

The solution for unknowns, therefore is \cite[p. 1178]{michalski2005electromagnetic}:
\begin{subequations}
  \begin{align}
    A_n = \frac{\overleftarrow{\Gamma_n} e^{j k_{zn} 2z_n }}
    {1-\overleftarrow{\Gamma_n} \overrightarrow{\Gamma_n} e^{-2j k_{zn} d_{n} }}
    \left[ e^{-j k_{zn} z'} + \overrightarrow{\Gamma_n} e^{-j k_{zn} (2z_{n+1} -z')} \right],\\
    %
    B_n = \frac{\overrightarrow{\Gamma_n} e^{-j k_{zn} 2z_{n+1} }}
    {1-\overleftarrow{\Gamma_n} \overrightarrow{\Gamma_n} e^{-2j k_{zn} d_{n} }}
    \left[ e^{+j k_{zn} z'} + \overleftarrow{\Gamma_n} e^{+j k_{zn} (2z_n -z')} \right]
    \label{eq:AnB}
  \end{align}
\end{subequations}
where $\overleftarrow{\Gamma_n}$ and $\overrightarrow{\Gamma_n}$ are the left and right looking reflection coefficients:

\begin{subequations}
  \begin{align}
  \overleftarrow{\Gamma_n} ={}& \frac{\overleftarrow{Z_n} - Z_n}{\overleftarrow{Z_n} + Z_n}
    \label{eq:R_left}\\
    %
    \overrightarrow{\Gamma_n} ={}& \frac{\overrightarrow{Z_n} - Z_n}{\overrightarrow{Z_n} + Z_n}
      \label{eq:R_right}
  \end{align}
  \label{eq:R}
\end{subequations}

and $d_n = z_{n+1} - z_n$ is the thickness of the $n^{th}$ section of the transmission line as illustrated in Fig.()






  \clearpage % Force Bibliography to the end of document
  \bibliography{mylib}
  \bibliographystyle{ieeetr}

\end{document}
